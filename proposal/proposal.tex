%
% File acl2020.tex
%
%% Based on the style files for ACL 2020, which were
%% Based on the style files for ACL 2018, NAACL 2018/19, which were
%% Based on the style files for ACL-2015, with some improvements
%%  taken from the NAACL-2016 style
%% Based on the style files for ACL-2014, which were, in turn,
%% based on ACL-2013, ACL-2012, ACL-2011, ACL-2010, ACL-IJCNLP-2009,
%% EACL-2009, IJCNLP-2008...
%% Based on the style files for EACL 2006 by 
%%e.agirre@ehu.es or Sergi.Balari@uab.es
%% and that of ACL 08 by Joakim Nivre and Noah Smith

\documentclass[11pt,a4paper]{article}
\usepackage[hyperref]{acl2020}
\usepackage{times}
\usepackage{latexsym}
\usepackage{multirow}
\renewcommand{\UrlFont}{\ttfamily\small}

% This is not strictly necessary, and may be commented out,
% but it will improve the layout of the manuscript,
% and will typically save some space.
\usepackage{microtype}

\aclfinalcopy % Uncomment this line for the final submission
%\def\aclpaperid{***} %  Enter the acl Paper ID here

%\setlength\titlebox{5cm}
% You can expand the titlebox if you need extra space
% to show all the authors. Please do not make the titlebox
% smaller than 5cm (the original size); we will check this
% in the camera-ready version and ask you to change it back.

\newcommand\BibTeX{B\textsc{ib}\TeX}

\title{Proposal for Analyzing Language Models: Separatablity of Syntax and Semantics}

\author{Qingxia Guo, Saiya Karamali, Lindsay Skinner, \and Gladys Wang
 \\ University of Washington \\ 
\texttt{\{qg07, karamali, skinnel, qinyanw\}@uw.edu}\\ 
}
\date{}

\begin{document}

\maketitle

\section{Introduction}

The boundary between semantics and syntax has always been a hotly debated topic in linguistics, but do NLP models make such a distinction? The objective of this project is to explore BERT\rq s \citep{bert} reliance on certain syntactic information when handling a semantic task, and vice versa. To achieve our goal, we will construct a linear probing system for a task and then employ Iterative Nullspace Projection (INLP from here on) \citep{inlp} to generate a new embedding devoid of information learned from the probing task. Then we will measure the performance of this new embedding on various tasks.

The design of our probing procedure follows \citealp{amnesia}, who employed INLP to investigate whether BERT uses part-of-speech (POS) information when solving language modeling (LM) tasks. A novel method for removing information from a embedding, INLP iteratively trains a linear model on a specific task, and projects the input on the nullspace of the linear model. Our objective is that, by applying the INLP procedure to a syntactic task, we are able to separate the representation into a syntactic space and a non-syntactic space. Then we can compare the performance of a semantic task in the whole space and only in the non-syntactic space to see if BERT is using syntactic information when performing the semantic task. Conversely, we can also first probe a semantic task and then measure with a syntactic task. Once we derive the semantic space and syntactic space from the experiment, we can further conduct analysis on the separability of the two spaces. 

To evaluate the separability of syntax and semantics, we will need two task, one task for the probing system and INLP procedure, one task for evaluating performance on embeddings before and after INLP. We choose Combinatory Categorical Grammar (CCG from here on) tagging \citep{ccg-bank} as the syntactic task and Semantic Role Labeling (SRL from here on) \citep{propbank} as the semantic task. Section \ref{sec:method} provides description of the tasks and the construction of the experiment pipeline. Section \ref{sec:result} provides possible results and interpretations. Finally, section \ref{sec:management} provides project management details including timeline and division of labor.

\section{Methods}
\label{sec:method}

We construct two separate probing tasks to isolate the syntactic and semantic information in word-level BERT embeddings. The embeddings are separated into syntactic and non-syntactic, and semantic and non-semantic components via INLP which is described in section \ref{sec:inlp-method}. These embedding components are then combined to form new embeddings, which are evaluated on the same tasks that were used for probing. If time permits, we will also evaluate how these new embeddings perform on the language modeling task. 
%cite BERT paper maybe?
%cite INLP paper

\subsection{The Iterative Null-Space Projection method}
\label{sec:inlp-method}

The INLP is used to create a guarding function that masks all the linear information contained in a set of vectors, $X$, that can be used map each vector to $c \in C$, where $C$ is the set of all categories. This is accomplished by training a linear classifier, a matrix $W$, that is applied to each $x \in X$ in order to predict the correct category $c$ with the greatest possible accuracy. In other words, $Wx$ defines a distribution over the set of categories $C$ and we assign $x$ to the class $c \in C$ which is allotted the greatest probability by $Wx$. Note that the classifier's accuracy must be greater than random chance, otherwise $x$ contains no linear information relevant for the categorization task and thus no guarding function is needed. Once $W$ is determined, for any $x \in X$ we can remove the information that $W$ uses to predict $c$ by projecting $x$ onto the null-space of $W$, $N(W) = \{x | Wx=0\}$. Call this projection function $P_1$ and let $\hat{x} = P_1(x)$. This removes all of the linear information in $x$ that $W$ used to predict the category $c$. 
%Need to cite INLP paper

However, this process does not necessarily remove all of the linear information in $x$ that could be used to predict $c$. For example, $x$ may contain redundant information and $W$ may have only used one set of this information for its prediction. In this case, the redundant information would still be present in $\hat{x}$. Thus, we must repeat the above process, defining a new linear classifier $\hat{W}$ that uses $\hat{x}$  to predict $c$. If $\hat{W}$ is still able to predict $c$ with a greater than random chance accuracy, then we know that $\hat{x}$ contained linear information about $c$. As above, we project $\hat{x}$ onto the null-space of $\hat{W}$ via the projection function $P_2$ and define a new $\hat{x} = P_2(P_1(x))$.

We iteratively apply this process until no linear information remains in $\hat{x}$, i.e. a linear classifier is unable to predict the correct category $c$ with any probability greater than random chance. The final $\hat{x} = P_n(P_{n-1}(\dots P_1(x)))$ contains no linear information about the categories in $C$ and we call $P(x) = P_n(P_{n-1}(\dots P_1(x)))$ the guarding function. 

We will pair the INLP method with the probing tasks described in sections \ref{sec:syntactic} and \ref{sec:semantics} in order to create two guarding functions that will enable us to isolate components of BERT embeddings that contain syntax-specific and semantics-specific information. 

\subsection{Data}
\label{sec:data}

%Description of the data set used and the information it contains relevant to the tasks of interest. Talk about pre-processing the syntax trees to get CCG tags. 
We will use the English V4 subset of the CoNLL 2012 shared task data \citep{2012-conll} . We must perform two pre-processing steps for this data to be used for our probing and evaluation tasks. The first is to apply \citet{ccg-bank}'s CCG Derivation algorithm to the parse tree field in the dataset, in order to create the CCG tags for each word. If this proves to be too time-consuming or computationally expensive then we shall change the syntactic probing task to utilize the POS tags available in the dataset, in place of CCG tags. The second task is to use the SRL frames in the dataset to generate (verb, BIO-argument-tag) pairs that will act as the categories for the semantic probing task. 
%cite CCG paper

\subsection{Syntactic probing task}
\label{sec:syntactic}

The syntactic probing task involves training a linear classifier on the final layer BERT embeddings in order to predict the CCG tag associated with each word. We will use this classifier in the INLP algorithm in order to create a guarding function for the information that is necessary to complete the CCG labeling task. For a given embedding, the projection that results from applying this guarding function to the embedding will represent the non-syntactic information contained in the embedding and will from now on be referred to as the ``non-syntactic component'' of the embedding. We can then determine the ``syntactic component'' of the embedding by taking the difference of the embedding vector with the non-syntactic component. 

\subsection{Semantic probing task}
\label{sec:semantics}

Similar to the above, the semantic probing tasks involves training a linear classifier on the final layer BERT embeddings in order to predict the semantic role tag (described in the data section) associated with each word. This classifier is used in the INLP algorithm in order to create a guarding function for the information necessary to complete the SRL labeling task. For this particular task, it is possible that a single word will have multiple SRL tags associated with it. In this case, if the word has $n$ SRL tags associated with it then we will look at the $n$ most likely tags output by the linear classifier. We will then treat each (vector, SRL tag) pair as a single example when calculating the classifier's accuracy in order to create a guarding function that works across all of the SRL tags affiliated with a particular word. As described in the Syntactic probing task section, we shall use the resulting guarding function to decompose the original embedding into a ``semantic component'' and a ``non-semantic component''. 

\subsection{Evaluation tasks}
\label{sec:eval}

Our goal is to determine which information sets captured in the BERT embeddings are relevant for our evaluation tasks. We thus use the components derived from the probing tasks to create new embeddings that isolate specific types of information. These embeddings are then evaluated on the syntactic and semantic tasks that were used for probing, and their performance is compared to that of the original embeddings. 

The new embeddings to be tested include the syntactic component, the non-syntactic component, the semantic component and the non-semantic component derived from the probing tasks. Additionally, we can create an embedding that contains syntactic information and removes semantic information by linearly projecting the syntactic component onto the non-semantic component. Using a similar process, we can create an embedding that contains semantic information and removes the syntactic information present. Finally, we can create an embedding that contains the semantic information captured by the syntactic component, by linearly projecting the syntactic component onto the semantic component. Similarly, we can create an embedding that contains the syntactic information captured by the semantic component. 

We will assess each of these embedding types and the original BERT embeddings on the CCG and SRL labeling tasks that were used in the probes. We have also hypothesized several additional assessment tasks that we would like to undertake, if time permits, or relegate to future work. The first of these tasks is to assess how each embedding type performs on the language modeling task. We would also like to perform the evaluation classification task using a feed-forward neural network with a single hidden layer that contains 10 nodes, in order to determine if there is any task-relevant non-linear information present in the embeddings. If time permits, we would also like to look for patterns in the performance of different embeddings, e.g. explore if a particular embedding type tends to perform better/worse on one of the evaluation tasks for words of a particular POS compared to others. Finally, if time permits we would like to repeat the above procedure to explore the embeddings output by different layers of the BERT model. 
%FFNN description is entirely arbitrary, it was just something we talked about during one of our meetings. If anyone has a particular architecture in mind here, please feel free to update this section.




\section{Possible Results}
\label{sec:result}

In this section, we consider what each evaluation task tells us about the embeddings that are being probed. When we isolate the syntactic component and run it on the syntactic and semantic tasks, we learn how successfully the component responsible for CCG tagging has been isolated, and we also learn how effective the syntactic component alone is on the semantic task. Similarly, running the semantic component on the evaluation tasks tells us how well we've isolated the semantic component and how effective it is on the syntactic task. If the removing of syntactic information leads to an insignificant decrease in the model\rq s performance in a semantic task (or removing semantic information leads to insignificant decrease in performance in a syntactic task), we can conclude that syntactic and semantic information in BERT\rq s representation is linearly separable. Conversely, a significant decrease in performance will indicate low separability between syntactic and semantic information in BERT \rq s representation. Finally, running the non-syntactic and non-semantic components on the evaluation tasks tells us whether any information not identified by INLP is at all useful for the evaluation tasks.

Next, we consider the potential results of the various projected word embeddings on the evaluation tasks. Each of these tell us how much overlap there is between the syntactic and semantic components of the contextual word embeddings. Projecting the syntactic component onto the non-semantic component removes semantic information from the syntactic component, and projecting the semantic component onto the non-syntactic component removes syntactic information from the semantic component. Projecting the syntactic component onto the semantic component gives us the semantic information that is also part of the syntactic component, and projecting the semantic component onto the syntactic component gives us the syntactic information that is also part of the semantic component. Running all of these embeddings on the semantic and syntactic tasks tells us how separated the semantic and syntactic components are, and how important the overlapping portions are to each task.




\section{Division of Labor and Timeline}
\label{sec:management}
There are three main parts of this project, coding the experiment, presenting related topics in class and writing the final paper. Each part has detailed requirements along with due dates and been assigned to a member to take charge of as presented in table \ref{role description}. It is hard to anticipate the time consumed and difficulty of each task, so all the due dates are temporary. We have set up a regular weekly meeting to update the progress of each part jso that everyone is on the same page with the development of the project. 

\begin{table*}[h]
    \centering
    \begin{tabular}{llll}
    \hline
    \textbf{Duty} & \textbf{Details} & \textbf{Who is in Charge} &\textbf{Due} \\
    \hline
    \multirow{5}{*}{Coding the experiment} & data preprocessing & Qingxia & 05/04 \\
    & building probing system \& INLP & Saiya \& Qingxia & 05/25 \\
    & building evaluation system & Gladys & 05/25 \\
    & merging subparts together & Saiya & 06/02\\ 
    & running different task pairs & Saiya & 06/02\\
    \hline
    \multirow{2}{*}{Presenting related topic} & \multirow{2}{*}{presenting INLP method} & \multirow{2}{*}{Lindsay \& Gladys} & \multirow{2}{*}{05/11}\\
    & & & \\
    \hline
    \multirow{2}{*}{Writing paper} & \multirow{2}{*}{writing out experiment } & \multirow{2}{*}{Lindsay} & \multirow{2}{*}{06/09}  \\
    & & & \\
    \hline
    \end{tabular}
    \caption{\label{role description} Role Description
    }
    \end{table*}






\bibliography{acl2020}
\bibliographystyle{acl_natbib}


\end{document}
