%
% File acl2020.tex
%
%% Based on the style files for ACL 2020, which were
%% Based on the style files for ACL 2018, NAACL 2018/19, which were
%% Based on the style files for ACL-2015, with some improvements
%%  taken from the NAACL-2016 style
%% Based on the style files for ACL-2014, which were, in turn,
%% based on ACL-2013, ACL-2012, ACL-2011, ACL-2010, ACL-IJCNLP-2009,
%% EACL-2009, IJCNLP-2008...
%% Based on the style files for EACL 2006 by 
%%e.agirre@ehu.es or Sergi.Balari@uab.es
%% and that of ACL 08 by Joakim Nivre and Noah Smith

\documentclass[11pt,a4paper]{article}
\usepackage[hyperref]{acl2020}
\usepackage{times}
\usepackage{latexsym}
\renewcommand{\UrlFont}{\ttfamily\small}

% This is not strictly necessary, and may be commented out,
% but it will improve the layout of the manuscript,
% and will typically save some space.
\usepackage{microtype}

\aclfinalcopy % Uncomment this line for the final submission
%\def\aclpaperid{***} %  Enter the acl Paper ID here

%\setlength\titlebox{5cm}
% You can expand the titlebox if you need extra space
% to show all the authors. Please do not make the titlebox
% smaller than 5cm (the original size); we will check this
% in the camera-ready version and ask you to change it back.

\newcommand\BibTeX{B\textsc{ib}\TeX}

\title{Proposal for Analyzing Language Models: Separatablity of Syntax and Semantics}

\author{Qingxia Guo \and Saiya Karamali \and Lindsay Skinner \and Gladys Wang
 \\ University of Washington \\ 
\texttt{qg07@uw.edu}\\ \texttt{karamali@uw.edu}  \\ \texttt{skinnel@uw.edu} \\ \texttt{@uw.edu}}

\date{}

\begin{document}

\maketitle
\begin{abstract}
This document contains the instructions for preparing a manuscript for the proceedings of ACL 2020.
The document itself conforms to its own specifications, and is therefore an example of what your manuscript should look like.
These instructions should be used for both papers submitted for review and for final versions of accepted papers.
Authors are asked to conform to all the directions reported in this document.
\end{abstract}




\section{Introduction}

The following instructions are directed to authors of papers submitted to ACL 2020 or accepted for publication in its proceedings.
All authors are required to adhere to these specifications.
Authors are required to provide a Portable Document Format (PDF) version of their papers.
\textbf{The proceedings are designed for printing on A4 paper.}


\section{Methods}


\section{Possible Results}
\label{sec:length}

The conference accepts submissions of long papers and short papers.
Long papers may consist of up to eight (8) pages of content plus unlimited pages for references.
Upon acceptance, final versions of long papers will be given one additional page -- up to nine (9) pages of content plus unlimited pages for references -- so that reviewers' comments can be taken into account.
Short papers may consist of up to four (4) pages of content, plus unlimited pages for references.
Upon acceptance, short papers will be given five (5) pages in the proceedings and unlimited pages for references. 
For both long and short papers, all illustrations and tables that are part of the main text must be accommodated within these page limits, observing the formatting instructions given in the present document.
Papers that do not conform to the specified length and formatting requirements are subject to be rejected without review.

The conference encourages the submission of additional material that is relevant to the reviewers but not an integral part of the paper.
There are two such types of material: appendices, which can be read, and non-readable supplementary materials, often data or code.
Additional material must be submitted as separate files, and must adhere to the same anonymity guidelines as the main paper.
The paper must be self-contained: it is optional for reviewers to look at the supplementary material.
Papers should not refer, for further detail, to documents, code or data resources that are not available to the reviewers.
Refer to Appendices~\ref{sec:appendix} and \ref{sec:supplemental} for further information. 

Workshop chairs may have different rules for allowed length and whether supplemental material is welcome.
As always, the respective call for papers is the authoritative source.


\section{Divsion of Labor and Timeline}
As reviewing will be double-blind, papers submitted for review should not include any author information (such as names or affiliations). Furthermore, self-references that reveal the author's identity, \emph{e.g.},
\begin{quote}
We previously showed \citep{Gusfield:97} \ldots
\end{quote}
should be avoided. Instead, use citations such as 
\begin{quote}
\citet{Gusfield:97} previously showed\ldots
\end{quote}
Please do not use anonymous citations and do not include acknowledgements.
\textbf{Papers that do not conform to these requirements may be rejected without review.}

Any preliminary non-archival versions of submitted papers should be listed in the submission form but not in the review version of the paper.
Reviewers are generally aware that authors may present preliminary versions of their work in other venues, but will not be provided the list of previous presentations from the submission form.

Once a paper has been accepted to the conference, the camera-ready version of the paper should include the author's names and affiliations, and is allowed to use self-references.

\paragraph{\LaTeX-specific details:}
For an anonymized submission, ensure that {\small\verb|\aclfinalcopy|} at the top of this document is commented out, and that you have filled in the paper ID number (assigned during the submission process on softconf) where {\small\verb|***|} appears in the {\small\verb|\def\aclpaperid{***}|} definition at the top of this document.
For a camera-ready submission, ensure that {\small\verb|\aclfinalcopy|} at the top of this document is not commented out.






\section*{Acknowledgments}

The acknowledgments should go immediately before the references. Do not number the acknowledgments section.
Do not include this section when submitting your paper for review.

\bibliography{anthology,acl2020}
\bibliographystyle{acl_natbib}

\appendix

\section{Appendices}
\label{sec:appendix}
Appendices are material that can be read, and include lemmas, formulas, proofs, and tables that are not critical to the reading and understanding of the paper. 
Appendices should be \textbf{uploaded as supplementary material} when submitting the paper for review.
Upon acceptance, the appendices come after the references, as shown here.

\paragraph{\LaTeX-specific details:}
Use {\small\verb|\appendix|} before any appendix section to switch the section numbering over to letters.


\section{Supplemental Material}
\label{sec:supplemental}
Submissions may include non-readable supplementary material used in the work and described in the paper.
Any accompanying software and/or data should include licenses and documentation of research review as appropriate.
Supplementary material may report preprocessing decisions, model parameters, and other details necessary for the replication of the experiments reported in the paper.
Seemingly small preprocessing decisions can sometimes make a large difference in performance, so it is crucial to record such decisions to precisely characterize state-of-the-art methods. 

Nonetheless, supplementary material should be supplementary (rather than central) to the paper.
\textbf{Submissions that misuse the supplementary material may be rejected without review.}
Supplementary material may include explanations or details of proofs or derivations that do not fit into the paper, lists of
features or feature templates, sample inputs and outputs for a system, pseudo-code or source code, and data.
(Source code and data should be separate uploads, rather than part of the paper).

The paper should not rely on the supplementary material: while the paper may refer to and cite the supplementary material and the supplementary material will be available to the reviewers, they will not be asked to review the supplementary material.

\end{document}
